\documentclass{article}
\usepackage{geometry}
\geometry{a4paper}
\usepackage{ctex}%中文
\usepackage{graphicx}%图片
\usepackage{multirow}%表格
\usepackage{amsmath}%公式,公式换行等时候会用到
\usepackage{amsmath}
\usepackage{amsfonts}%特殊符号,比如说实数集
\usepackage{subfigure} 
\title{最优化学习笔记附录——B3概率基础}
\author{BD S}
\begin{document}
\maketitle
\tableofcontents

\section{概率空间}
概率空间由一个三元组$(\Omega,F,P)$组成,其中$\Omega$是样本空间,$F$是事件域,以及在0-1之间的概率函数$P$。

\section{随机变量}
为了更好的描述随机现象量化的过程。
随机变量$X$是定义在样本空间$\Omega$上的实值\textbf{函数},即$X:\Omega \rightarrow \mathbb{R}$。
随机变量满足要求可测性,对于任意的$t \in \mathbb{R}$,集合$\{\omega| X(\omega)>t\}$总是属于$F$。

给定一个样本空间中的概率函数$P$,随机变量$X$大于某个值$x$的概率为$P(\omega|\omega(X)>x)$,简写为$P(X>x)$,最后的这句话是重点,结合到了所涉及到的知识点。
再具体一点,一个例子。样本空间设为$\Omega =\{1,2,\dots,8\}$,随机变量的定义为
\begin{equation}
 X(\Omega)=\left\{
    \begin{aligned}
    0, \omega \ is \ even  \\
    1, \omega \ is \ odd  
    \end{aligned}
\right.   
\end{equation}
就是奇数偶数的时候取值不一样。在这个例子中,我们感兴趣的就是使得$X$具有可测性的事件域中最小的那个,即$F=\{\emptyset ,\{1,3,5,7\},\{2,4,6,8\},\Omega\}$。个人理解就是根据要求来设定这个随机变量函数$X$,然后诱导出对应的事件域$F$。
\subsection{离散型随机变量和连续型随机变量}
首先讲一下分布函数。
可以知道,$P(a<X \le b)=P(x \le b)-P(x \le a)$。
定义一个随机变量的分布函数为$F(x)=P(X \le x), x \in \mathbb{R}$,此时也称$X$服从分布$F(x)$,记为$X\sim F(x)$,这是一个单调有界并且右连续的函数。

再讲一下概率密度函数。对于一个连续型变量$X$,如果存在函数$f(x)$,使得$$
F(x)=\int_{-\infty}^x f(t)dt
$$
有性质$$
\int_{-\infty}^{+\infty} f(t)dt=1
$$
对于连续型随机变量,每个点的概率密度都是0,$P(X=x)=0,\forall x \in \mathbb{R}$。
对于离散型随机变量,可列个点的概率密度为$p(x)=P(X=x)$
\subsection{数学期望}
对于离散型的随机变量$X$,设其取值的集合为${x_1,x_2,\dots}$,分布律为$P(x)$,若级数
$\sum\limits_{i=1}^{\infty}x_iP_i(x_i)$是绝对收敛的,则称其为$X$的数学期望。
\subsection{方差}
定义
$$
Var(X)=E[(X-E(X)^2)]
$$
同时也是
$$
Var(X)=E(X^2)-E(X)^2
$$
\subsection{联合分布}
上面讲的都是单个变量的情况,还有多个变量的情况。
比如对于两个随机变量$X$,$Y$,定义联合分布函数为$$
F(a,b)=P(X \le a,Y \le b),-\infty<a,b<+\infty
$$
通过联合分布也可以得到随机变量X的分布函数
$$
F_X(a)=P(X \le a)=P(X \le a,Y<+\infty)=\lim\limits_{b \rightarrow +\infty}F(a,b)
$$
对于这种只含部分分量的分布函数称为边缘分布函数。

以上是分布函数,下面是分布律。分布律定义为
$$p(x,y)=P(X=x,Y=y)$$
边缘分布律也可以类似计算
$$P_X(x)=\sum\limits_{y:p(x,y)>0}p(x,y)$$

接下来是联合密度函数$$P(X \in A,Y \in B)=\int_B\int_Af(x,y)dxdy$$
那么这个$f(x,y)$就说联合密度函数。边缘密度函数也类似,$f_X(x)=\int_{-\infty}^{+\infty}f(x,y)$

一个概念,相互独立。也就是
$$P(X \le a,Y \le b)=P(X \le a)P(Y \le b)$$
$$F(a,b)=F_X(a)F_Y(b)$$
对于离散情况
$$p(x,y)=p_X(x)p_Y(y)$$
对于连续情况
$$f(x,y)=f_X(x)f_Y(y)$$

对于两个变量不独立的情况,引入协方差。定义为$$
Cov(X,Y)=E((X-E(X)(Y-E(Y)))=E(XY)-E(X)E(Y)
$$
如果协方差是0,就说明相互独立。

本章概念,分布函数F、分布律p、边缘分布函数、边缘分布律(这四个差不多)
联合密度函数f、边缘密度函数。密度函数积分就是分布律,分布律积分就是分布函数。
都可以由概率函数P引出。
\section{条件期望}
假设现在有随机向量$(X,Y_1,Y_2,\dots,Y_n)$,并且已经观测到了$(Y_1,Y_2,\dots,Y_n)$,那么就要用$(Y_1,Y_2,\dots,Y_n)$来预测$X$。
\subsection{$L^2$空间}
设概率空间$(\Omega,F,P)$,定义$L^2$空间为$$
L^2(\Omega,F,P)=\{X|E(X^2)<+\infty\}
$$
是二阶矩有限的随机变量构成的集合。

考虑$L^2$空间,可以引入内积的概念。定义X与Y的内积$\left\langle X,Y \right\rangle=E[XY]$
引入范数和距离
$$
||X||_{L^2}=\sqrt{E[X^2]}
$$
$$
d(X,Y)=\sqrt{\left\langle E[(X-Y)^2] \right\rangle}
$$
\subsection{$L^2$空间上随机变量的条件期望}
\end{document}